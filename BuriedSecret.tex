%
% Niniejszy plik stanowi przyk³ad formatowania pracy magisterskiej na
% Wydziale MIM UW.  Szkielet u¿ytych poleceñ mo¿na wykorzystywaæ do
% woli, np. formatujac wlasna prace.
%
% Zawartosc merytoryczna stanowi oryginalnosiagniecie
% naukowosciowe Marcina Wolinskiego.  Wszelkie prawa zastrze¿one.
%
% Copyright (c) 2001 by Marcin Woliñski <M.Wolinski@gust.org.pl>
% Poprawki spowodowane zmianami przepisów - Marcin Szczuka, 1.10.2004
% Poprawki spowodowane zmianami przepisow i ujednolicenie 
% - Seweryn Kar³owicz, 05.05.2006

% dodaj opcjê [licencjacka] dla pracy licencjackiej
\documentclass{pracamgr}

\usepackage{polski}

\usepackage[utf8]{inputenc}

% Dane licencjanta:

\author	{Imię Nazwisko}

\nralbumu{666999}

\title{Implementacja taktycznej gry fabularnej czasu rzeczywistego}

\tytulang{An implementation of a real-time strategy role-playing game}

\kierunek{Informatyka}

\opiekun{mgra Radosława Bartosiaka\\
  Instytut Informatyki\\
  }

% miesiąc i rok:
\date{Maj 2015}

\dziedzina{11.3 Informatyka}

%Klasyfikacja tematyczna wedlug ACM (informatyka)
\klasyfikacja{D. Software}

\keywords{gra, gra komputerowa, qt, sfml, box2d, taktyczna gra fabularna}

% Tu jest dobre miejsce na Twoje własne makra i środowiska:
\newtheorem{defi}{Definicja}[section]

% koniec definicji

\begin{document}
\maketitle

%tu idzie streszczenie na strone poczatkowa
\begin{abstract}
  Niniejsza praca opisuje taktyczną grę fabularną stworzoną
  w ramach przedmiotu Zespołowy Projekt Programistyczny.
  W szczególności opisane zostało projektowanie silnika,
  mechaniki oraz interfejsu gry. Praca zawiera również 
  opis osiagniętego rezultatu wraz z wynikiem przeprowadzonych testów.
\end{abstract}

\tableofcontents
%\listoffigures
%\listoftables

\chapter*{Wprowadzenie}
\addcontentsline{toc}{chapter}{Wprowadzenie}


\chapter{Podstawowe pojęcia}

\chapter{Wstęp}

  \section{Opis projektu}

  \section{Grupa docelowa}
 
  \section{Analiza podobnych gier}
 
  \section{Mechanika}
  
  \section{Fabuła}
  
  \section{Interfejs}

\chapter{Narzędzia i metodologia pracy}

  \section{Użyte wzorce projektowe}

  \section{Użyte narzędzia}

  \section{Zasady prowadzenia playtestów}
  
  \section{Edytor treści}

\chapter{Kamienie milowe}

  \section{Podstawowy silnik gry (Wersja 0.1)}
  
  \section{Ukończenie przewidzianych funkcjonalności gry (Wersja 0.2)}
  
  \section{Gotowy produkt (Wersja 0.3)}

\chapter{Wkład własny w powstały system}

  \section{Jan Darowski}

  \section{Piotr Majcherczyk}

  \section{Rafał Soszyński}

  \section{Tomasz Zakrzewski}

\chapter{Rezultat}

  \section{Zrealizowane funkcjonalności}

  \section{Niezrealizowane funkcjonalności}

  \section{Opinie testerów}

  \section{Podsumowanie autorów}

\appendix

\chapter{Przykładowe zrzuty ekranu}

\chapter{Pomoce do przeprowadzania testów}

\chapter{Zawarość płyty CD}


\begin{thebibliography}{99}
\addcontentsline{toc}{chapter}{Bibliografia}

\item{tu wpisać bibliografię}

\end{thebibliography}

\end{document}


%%% Local Variables:
%%% mode: latex
%%% TeX-master: t
%%% coding: latin-2
%%% End:
